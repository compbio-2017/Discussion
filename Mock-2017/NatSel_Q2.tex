In the simplest selection model consisting of a single locus and two candidate alleles, many details are overlooked, as selection often occurs over multiple loci, each of which have multiple (i.e. more than two) candidate alleles.

When selection occurs over multiple loci in a haploid species, we begin to see phenomena such as clonal interference, i.e. when a beneficial allele is out-competed by another and genetic hitchhiking, i.e. when a beneficial allele carries a deleterious allele to fixation. Genes often interact with each other (epistasis) and fitness is not necessarily additive. We may also get linkage disequilibrium (especially in diploid individuals, where recombination is common), where there is non-random association of alleles at two or more loci.

Within a species, there also may be frequency-dependent selection, where the fitness of an individual is dependent on the proportion of other individuals in the population that are genetically similar.

Beyond selection, the simple model fails to account for other forms of genetic variation such as mutation and genetic drift.
